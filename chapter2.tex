\chapter{Розробка шаблону для студентських накових робіт засобами видавничої системи \LaTeX{}} 
\label{chap:second}

\section{Першочергові задачі з розробки шаблону}

\subsection{Титульний аркуш}

Стандарт наводить серію вимог \cite[с.~15]{DSTU20153008} щодо змісту титульного аркушу та взаємного розміщення окремих елементів, наприклад порядку реквізитів або підписів відповідальних осіб. Проте, маже не наводиться суттєвих вимог до форматування цієї частини документу. Тому було прийнято рішення взяти за основу шаблону титульного аркуша зразок з документу <<Методичні рекомендації до підготовки курсових та випускних робіт для студентів економічних спеціальностей>> \cite[c.~39]{doc:methodika:1}.

Для максимального розділення команд форматування титульного аркушу і його змісту, що вноситиметься автором роботи, все формування титульного аркушу винесене в окремий макрос, що отримує один параметр\,--\,список пар <<назва параметру>>\,--\,<<значення>> (остайнє реалізовано в пакеті xkeyval).

Команда виклику макросу, в свою чергу, винесена в окремий файл, що підключається до основного файлу роботи. Крім іншого, це відділення дозволить автору, одноразово заповнивши параметри, не повертатися до них без необхідності.

\subsection{Форматування елементів рубрикації та змісту}
\label{dev:toc}

Переопределяем все!

\subsection{Форматування тексту та формул}
В виключеній формулі \ref{eq:gaussianIntegral} показано використання деяких математичних операцій, а також посилань на формули.

\begin{equation}
\label{eq:gaussianIntegral}
\int\limits^{+\infty}_{-\infty} e^{-x^2} dx = \sqrt{\pi} 
\end{equation}

\subsection{Форматування зображень та таблиць}
Там все непросто.

\subsection{Форматування додатків}

Сложно.

\section{Додаткові задачі з покращення шаблону}
