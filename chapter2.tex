\chapter{Розробка шаблону для студентських накових робіт засобами видавничої системи \LaTeX{}} 
\label{chap:second}

\section{Першочергові задачі з розробки шаблону}

\subsection{Титульний аркуш}

Стандарт наводить серію вимог \cite[с.~15]{DSTU20153008} щодо змісту титульного аркушу та взаємного розміщення окремих елементів, наприклад порядку реквізитів або підписів відповідальних осіб. Проте, маже не наводиться суттєвих вимог до форматування цієї частини документу. Тому було прийнято рішення взяти за основу шаблону титульного аркуша зразок з документу <<Методичні рекомендації до підготовки курсових та випускних робіт для студентів економічних спеціальностей>> \cite[c.~39]{doc:methodika:1}.

Для максимального розділення команд форматування титульного аркушу і його змісту, що вноситиметься автором роботи, все формування титульного аркушу винесене в окремий макрос, що отримує один параметр\,--\,список пар <<назва параметру>>\,--\,<<значення>> (остайнє реалізовано в пакеті xkeyval).

Команда виклику макросу, в свою чергу, винесена в окремий файл, що підключається до основного файлу роботи. Крім іншого, це відділення дозволить автору, одноразово заповнивши параметри, не повертатися до них без необхідності.

\subsection{Форматування елементів рубрикації та змісту}
\label{dev:toc}

В стандарті чітко описано вимоги щодо оформлення заголовків структурних частин документу та тексту а також їх відображення у змісті \cite[с.~4]{DSTU20153008}\cite[с.~8]{DSTU20153008}.

Форматування, що пропонується \LaTeX{} за замовчуванням не відповідає наведеним вимогом. Для зміни оформлення змісту та заголовків, серед багатьох пакетів, обрано titletoc та titlesec відповідно.

Можливості пакетів надзвичайно широкі, проте для задоволення потреб було достатньо лише одного макросу з кожного, <<$\backslash$>>titlecontents для зміни форматування елементів змісту та <<$\backslash$>>titleformat для зміни форматування заголовків.

\subsection{Форматування тексту та формул}

\LaTeX{} відмінно справляється з форматуванням тексту. Для відповідності стандарту було внесено лише декілька змін, серед яких розміри полів документу (використано пакет geometry), налаштування колонтитулів, а саме~--- нумерація сторінок (пакет fancyhdr) та параметри шрифту (кегль, міжрядковий інтервал та абзацні відступи). Для додавання абзацного відступу для перших абзаців розділу (на притивагу зарубіжній традиції) підключено пакет indentfirst.

Робота з формулами не вимагає додаткових налаштувань. Виключенням можуть бути, наприклад, використання специфічних символів або їх накреслення чи коректування відстані між частинами деяких формул, що досягається підключенням пакетів. Вирішення подібних специфічних питань та пошук необхідних пакетів залишено на користувачів шаблону.

В виключеній формулі \ref{eq:gaussianIntegral} показано використання деяких математичних операцій, а також посилань на формули.

\begin{equation}
\label{eq:gaussianIntegral}
\int\limits^{+\infty}_{-\infty} e^{-x^2} dx = \sqrt{\pi} 
\end{equation}

\subsection{Форматування зображень та таблиць}
Там все не очень просто.

\subsection{Форматування додатків}

Сложно. 

\section{Додаткові задачі з покращення шаблону}

Пофиксить отступ в подписи таблицы. И Додаткы.
