\chapter{Розробка шаблону для студентських накових робіт засобами видавничої системи \LaTeX{}} 
\label{chap:second}

\section{Першочергові задачі з розробки шаблону}

\subsection{Титульний аркуш}

Стандарт наводить серію вимог \cite[с.~15]{DSTU20153008} щодо змісту титульного аркушу та взаємного розміщення окремих елементів, наприклад порядку реквізитів або підписів відповідальних осіб. Проте, маже не наводиться суттєвих вимог до форматування цієї частини документу. Тому було прийнято рішення взяти за основу шаблону титульного аркуша зразок з документу <<Методичні рекомендації до підготовки курсових та випускних робіт для студентів економічних спеціальностей>> \cite[c.~39]{doc:methodika:1}.

Для максимального розділення команд форматування титульного аркушу і його змісту, що вноситиметься автором роботи, все формування титульного аркушу винесене в окремий макрос, що отримує один параметр\,--\,список пар <<назва параметру>>\,--\,<<значення>> (остайнє реалізовано в пакеті xkeyval).

Команда виклику макросу, в свою чергу, винесена в окремий файл, що підключається до основного файлу роботи. Крім іншого, це відділення дозволить автору, одноразово заповнивши параметри, не повертатися до них без необхідності.

\subsection{Форматування елементів рубрикації та змісту}
\label{dev:toc}

В стандарті чітко описано вимоги щодо оформлення заголовків структурних частин документу та тексту а також їх відображення у змісті \cite[с.~4]{DSTU20153008}\cite[с.~8]{DSTU20153008}.

Форматування, що пропонується \LaTeX{} за замовчуванням не відповідає наведеним вимогам. Для зміни оформлення змісту та заголовків, серед багатьох пакетів, обрано titletoc та titlesec відповідно.

Можливості пакетів надзвичайно широкі, проте для задоволення потреб було достатньо лише одного макросу з кожного, $\backslash$titlecontents для зміни форматування елементів змісту та $\backslash$titleformat для зміни форматування заголовків.

Вимоги стандарту щодо оформлення структурної частини <<Зміст>> детально описано і зображено на прикладі, аналогічно методичним рекомендаціям. Проте, між ними знайдено деякі протиріччя, перечислені нижче.
\begin{enumerate}
\item Присутність напису <<Розділ>> перед назвою розділу.
\item Необхідність включення до змісту заголовків рівня <<пункт>>.
\item Стильове виділення записів рівня <<розділ>>.
\end{enumerate}

Розглянувши вищеперечислені пункти детальніше досягнуто наступних висновків:
\begin{enumerate}
\item На відміну від всіх методичних рекомендації в стандарті відсутня назва рівня розділу перед його номером і назвою. Проте підписано рівень <<Частина>>, найвищий у наведеному прикладі. Зважаючи на те, що у студентських наукових роботах найвищим рівнем рубрикації документу є <<Розділ>>, прийнято рішення про доцільність виділення його таким чином.
\item Не зважаючи на відсутність у наведених прикладах змістів документів заголовків пунктів, їх включення передбачено як вимогами стандарту, так і методичними рекомендаціями. При цьому, за необхідності, автор вільно може змінити кількість рівнів заголовків у змісті зміною значення лічильника tocdepth у преамбулі документу.
\item Стандартом не оговорено вимог щоди виділення будь"=яких частин змісту або його відсутності. Слідуючи прийнятому в першому пункті цього спису рішення та зразкам з методичних рекомендацій записи рівню <<розділ>> додатково виділено напівжирним шрифтом.
\end{enumerate}

\subsection{Форматування тексту та формул}

\LaTeX{} відмінно справляється з форматуванням тексту. Для відповідності стандарту було внесено лише декілька змін, серед яких розміри полів документу (використано пакет geometry), налаштування колонтитулів, а саме~--- нумерація сторінок (пакет fancyhdr) та параметри шрифту (кегль, міжрядковий інтервал та абзацні відступи). Для додавання абзацного відступу для перших абзаців розділу (на притивагу зарубіжній традиції) підключено пакет indentfirst.

Робота з формулами не вимагає додаткових налаштувань. Виключенням можуть бути, наприклад, використання специфічних символів або їх накреслення чи коректування відстані між частинами деяких формул, що досягається підключенням пакетів. Вирішення подібних специфічних питань та пошук необхідних пакетів залишено на користувачів шаблону.

В виключеній формулі \ref{eq:gaussianIntegral} показано використання деяких математичних операцій а також посилань на формули.

\begin{equation}
\label{eq:gaussianIntegral}
\int\limits^{+\infty}_{-\infty} e^{-x^2} dx = \sqrt{\pi} 
\end{equation}

\subsection{Форматування зображень та таблиць}
\label{figuresAndTablesFormat}
Для додавання зображень документу використано пакет graphicx з драйвером pdftex. Остайній дозволяє доавати до документу зображення в популярних форматах png та jpg. При цьому неможливо використовувати eps графіку та не  гарантується коректність конвернування файлу документу з формату pdf в rtf для подальшого редактування в текстових процесорах. В силу того, що \LaTeX{} цілком задовільняє потреби у оформленні документу, забезпечення цієї можливості на входить в завдання по розробці шаблону.

На противагу методичним рекомендаціям до оформлення курсових робіт у Херсонському державному університеті \cite[с.~18]{doc:methodika:1} \cite{doc:methodika:2}, які вимагають робити підпис над таблицею вирівняним по правій стороні і курсивним шрифтом, стандарт вимагає \cite[с.~10]{DSTU20153008} робити підпис однією фразою і з абзацного відступу. Було прийнято рішення про переважність державного стандарту над методичними рекомендаціями університету.

\subsection{Форматування додатків}
\label{appendixFormat}
Однією з структурних частин документу є додатки. Вони розміщуються після всього документу і слугують для детального розгляду окремих питань, включення яких до основної частини не є доцільним через іх об'єм або спосіб відтворення. В стандарті наведено специфічні вимоги щодо нумерації додатків, яких важко досягти базовими засобами \LaTeX{}, через що прийнято рішення про відкладення реалізації цього функціоналу.

\section{Додаткові задачі з покращення шаблону}

Для повного досягнення рішень, описаних в пункті \ref{figuresAndTablesFormat} необхідне більш детальне вивчення документації пакету caption, що дозволяю змінювати форматування заголовків або їх частин у таблись та рисунків, включених до документу.

Додатки являються необхідною структурною частиною більшості оформлених результатів наукових досліджень. Забезпечення повноцінного функціонування, частково описане в пункті \ref{appendixFormat} є однією з ключових частих майбутньої роботи над шаблоном.

Крім додатків, до наукової роботи входять або можуть входити за бажанням автора інші структурні частини, серед яких <<Вступ>>, <<Реферат>>, <<Висновки>>, <<Рекомендації>>. Відповідно до вимог, вони входять до Змісту на рівні з розділами, проте ними не являються. В \LaTeX{} не передбачено такого рівня рубрикації, тож його створення і налаштування входить до задач з подальшого розвитку шаблону.

При верстанні документу може скластися ситуація, за якої декілька остайніх строк тексту розділу або структурної частини можуть бути перенесені на наступну сторінку, що являється візуально недоречним. На даний момент для запобігання цій ситуації не прийнято жодних дій, попри це, в \LaTeX{} входять потужні механізми керування переносами шляхом додавання <<клею>> (в пояняттях \TeX{}~--- заповнювач між атомарними блоками, що може стискатись і розтягуватись) між словами і строками. Прийнято рішення про теоретичну можливість використання цих механізмів для запобігання описаній ситуації.
