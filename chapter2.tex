\chapter{Розробка шаблону для студентських накових робіт засобами видавничої системи \LaTeX{}} 
\label{chap:second}

\section{Першочергові задачі з розробки шаблону}

\subsection{Титульний аркуш}

Почему так сложно, что вообще нужно.

\subsection{Форматування елементів рубрикації та змісту}

Переопределяем все!

\subsection{Форматування тексту}
Абзаци, поля -- інше по дефолту.

\subsection{Форматування формул}
В виключеній формулі \ref{eq:gaussianIntegral} показано використання деяких математичних операцій, а також посилань на формули.

\begin{equation}
\label{eq:gaussianIntegral}
\int\limits^{+\infty}_{-\infty} e^{-x^2} dx = \sqrt{\pi} 
\end{equation}

\subsection{Форматування таблиць}
Там все непросто.

\subsection{Форматування зображень}

Сложно.

\subsection{Форматування додатків}

Сложно.

\section{Додаткові задачі з покращення шаблону}

\subsection{Что-то}
