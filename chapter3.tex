\chapter{Підготовка демонстраційних прикладів та перспективи використання шаблону} 
\label{chap:first}

\section{Використання \LaTeX{} при оформленні робіт}

\subsection{Практика використання \LaTeX{} в науковій сфері}

Система \LaTeX{} у всьому світі широко використовується науковцями, особливо фахівцями з фундаментальних наук.

Основне призначення системи~--- подготовка наукових документів (як правило, з технічних і фізико"=математичних наук). \LaTeX{} зручно використовувати для підготовки звітів з великою кількістю формул, таблиць та великим списко використаної літератури. Більшість наукових видань приймають тексти в форматі tex.

\subsection{Використання \LaTeX{} в ВНЗ}

Для фізико"=математичних спеціальностей актуальним є освоєння спеціалізованого програмного забезпечення для оформлення результатів дослідження. 

Крім того, що \LaTeX{} надає студентам потужний інструмент для оформлення результатів власних наукових робіт, він розвиває абстрактне мислення завдяки якістно відміному від загальноприйнятого у текстових процесорах WYSIWYG"=підходу до роботи з документом, адже написання и оформлення тексту і перегляд результату~--- це різні операції. Ще однією побічною перевагою використання \LaTeX{} у повсякденный роботі є вивчення базових прийомів програмування при написанні і використанні макросів.

В реаліях, де \LaTeX{} являється стандартом де"=факто для публікацій в технічних і фізико"=математичних журналах, його використання в вищих навчальних закладах при оформленні результатів студентських наукових робіт має підготувати майбутнього науковця до масштабної наукової роботи у майбутньому. В Україні численні внз вже рекомендують \LaTeX{} для оформлення курсових та випускних робіт. Серед них має сенс виділити національний технічний університет України <<Київський політехнічний інститут імені Ігоря Сікорського>>, де на кафедрі прикладної математики \LaTeX{} використовується для оформлення звітів до лабораторних робіт а на сайті кафедри електромеханічного обладнання енергоємних виробництв, крім того, для навчання студентів розміщені теоретичні та практичні завдання по роботі з пакетом \LaTeX{}.

\subsection{Переваги використання розробленого шаблону}

Створений шаблон дозволяє, з однієї сторони, полегшити роботу студентів на завершальному етапі проведення досліджень, а саме —  підготовці і оформленні їх результаті, а з іншої~--- стандартизувати оформлення наукових робіт в межах університету та гарантувати їхню відповідність Державному стандарту.

Відповідно до підходу \LaTeX{}, текст документу і його оформлення~--- це різні і незалежні частини. Таким чином, протягом підготовки і оформлення результатів наукової роботи, перед користувачем шаблону не стоїть задача слідкувати за зовнішнім виглядом документу. За умови коректної розмітки документу ця задача виконується самим \LaTeX{} згідно правил, попередньо заданих у шаблоні.

Ще однією перевагою розділення тексту і форматування документу є можливість миттєвої зміни оформлення документу шляхом підключення іншого файлу"=шаблону, яка широко використовується при надсиланні матеріалів наукової роботи до різних наукових журналів, кожен з яких може надавати свій власний стиль.

\section{Розробка інструкції та наочних прикладів по використанню шаблону}

З ціллю забезпечення вільного розуміння структури шаблону користувачами та можливості безопірного його редагування код шаблону повністю прокоментовано. На рисунку \ref{ris:tmpBegining} зображено заголовок шаблону, що являється новим типом документу.

\begin{figure}[h]
\center{\includegraphics[width=1\linewidth]{img/tmpBegining}}
\caption{Заголовок макросу}
\label{ris:tmpBegining}
\end{figure}

Для роботи над підготовкою результатів наукової роботи має сенс у користуванні системою контролю версій. Це позбавляє необхідності у створенні тимчасових або постійних копій частин документу і дозволяє легко повернутися до будь"=якої попередньої версії. Для розміщення шаблону обрано популярний веб"=сервіс github.com. Він працює з системою контролю версій git та дозволяє вільно ділитися результатами роботи з іншими користувачами. 

Шаблон з стильовим файлом та базовою структурою проекту документу розміщено на веб"=сервісі github.com за адресою https://github.com/sen-den/latex-template/tree/publicTemplate. Для початку роботи з шаблоном слід встановити систему контролю версій git та завантажити шаблон на локальний комп'ютер. Якщо встановлення додаткового програмного забезпечення не є доцільним, достатньо завантажити файли з розширенням tex, cls, bib та папку img. Для роботи над проектом без зразку достатньо завантажити файл khsu.cls (власне шаблон) та підключити його до основного файлу проекту.

\subsection{Оформлення титульної сторінки і рубрикація документу}

Для спрощення керування вмістом титульної сторінки він повністю відділений від керування його оформлення. Виклик команди"=макросу, що формує титульну сторіну розміщено в окремому файлі title.tex. Для формування власного титульного користувачеві необхілно і достатньо відкоригувати цей файл, замінивши значення за замовчуванням власними.

Сліз звернути увагу на те, що в силу особливостей роботи \LaTeX{}, не у кожний макрос, що приймає параметром текст можна передати текст з команодою нового абзацу. В даному макросі це прийнято до уваги і за передбаченого заповнення помилок чи переповнення сторінки не відбудеться.

Державним стандартом передбачено \cite[с.~15]{DSTU20153008} можливість розміщення титульного аркушу на двох і більше аркушах, проте через неможливість досягнення умов для цього при оформленні студентських робіт, на даний момент забезпечення такої можливості не є пріорітетним.

Слід звернути увагу на поле StudentSex макросу, яке відповідає за рід студента у підписі титульного аркушу. Очікується, що до нього буде введено <<male>> або <<female>> для виконавця чоловічого або жіночого роду відповідно. При цьому, при форматуванні цієї частини титульного аркушу використовується команда $\backslash$ifthenelse, яка реалізує умовний оператор. Ця команда має наступний синтаксис:
\begin{verbatim}
\ifthenelse{<умова>}{<перший блок команд>}{<другий блок команд>}
\end{verbatim}

За істинної умови макрос підставляє в точку свого виклику перший блок команд (в тому числі текст без виклику макросів та пустий блок), за хибної~--- другий.

Для рубрикації документу використовуються наступні команди:
\begin{verbatim}
\chapter{Назва розділу}
\section{Назва підрозділу}
\subsection{Назва пункту}
\subsubsection{Назва підпункту}
\paragraph{Назва параграфу}
\subparagraph{Назва підпараграфу}
\end{verbatim}

Перші три з них включають заголовок до змісту. Якщо необхідно включити до змісту заголовок, відмінний від заголовку в тілі документу, його необхідно вказати необов'язковим параметром перед ним, наприклад, так:
\begin{verbatim}
\chapter[Заголовок в змісті]{Заголовок в документі}
\end{verbatim}

Якщо ж заголовок не слід включати до змісту, то необхідно використати версію команди <<з зірочкою>>:
\begin{verbatim}
\chapter*{Заголовок, що не відобразиться в змісті}
\end{verbatim}

Керувати рівнем заголовків, які входитимуть до змісто можна змінюючи значення лічильника tocdepth у преамбулі документа. Рівень, до якого нумеруватимуться заголовки можна змінити, відповідно, лічильником secnumdepth.

\subsection{Форматування тексту та математичний режим}

Основною структурною частиною тексту є абзац, який створюється командою $\backslash$par або однією порожньою строкою. Дві і більше пустих строк не створять порожній абзац, що можливо у текстових процесорах.

Команди форматування тексту умовно можна поділити на три частини. 

Перші діють на текст, переданий їй в фігурних дужках, наприклад, команди виділення тексту напівжирним та курсивним шрифтом, дію яких вже було продемонстровано на рисунку \ref{ris:image}:
\begin{verbatim}
\textbf{<текст>}
\textit{<текст>}
\end{verbatim}

Другі починають діяти з моменту виклику і до кінця області бачення або виклику протилежної команди, наприклад
\begin{verbatim}
\bfseries
\end{verbatim}
матиме дію виділення напівжирним за описаним механізмом.

Треті є оточеннями, особливим видом макросів, дія яких не розглядатиметься окремо. Від звичайних макросів відрізняються тим, що додають певні вказані команди перед і після заключеного в них тексту. Одними з часто використованих оточень є оточення для маркерованих та нумерованих списків itemize та enumerate. Нижче буде розглянуто ще одне оточення для формування нумерованих формул.

Для включення до документу формул передбачено математичний режим, в якому \LaTeX{} знаходится між наступними командами в тексті документу:
\begin{verbatim}
$...$ 
\[...\]
\begin{equation}...\end{equation}
\end{verbatim}

Першою командою формулу, що знаходиться на місці трьох крапок буде розміщено в строці, другою~--- винесено с помыж строк, третьою~--- винесено та пронумеровано.

Нижче наведено зразок коду, що формує формулу \ref{eq:gaussianIntegral}. Команда $\backslash$label дозволяє посилатися на формулу за наведеним підписом командою $\backslash$ref\{<підпис>\}.
\begin{verbatim}
\begin{equation}
\label{eq:gaussianIntegral}
\int\limits^{+\infty}_{-\infty} e^{-x^2} dx = \sqrt{\pi} 
\end{equation}
\end{verbatim}

\subsection{Додавання таблиць та рисунків}

Для додавання таблиць підключено пакет xtable, що надає потужні багатосторінкові таблиці та перевизначено оточення table з ціллю максимального спрощення форматування таблиць.

Синтаксис оточення наступний:
\begin{verbatim}
\begin{table}
	{<Структура стовбців у звичайному форматі>}
	{<Заголовок таблиці>}
	{<Мітка таблиці>}
	{<Строка-заголовки стовбців>}
	<Таблиця в звичайній розмітці>
\end{table}
\end{verbatim}

Стосовно форматування таблиць наявна незакрита пріорітетна проблема, описана в пункті \ref{templateUpgrade}.

Для додавання ілюстрацій залишено незмінним стандартне оточення \LaTeX{} figure, яке достатньо задовільняє вимоги. Нижче наведено його синтаксис:
\begin{verbatim}
\begin{figure}
[<Команда пріорітетності розміщення в цому місці>]
\center{
	\includegraphics
		[<Розміри та параметри рисунку>]
		{<Шлях до рисунку>}
	}
\caption{<Заголовок рисунку>}
\label{<Мітка рисунку>}
\end{figure}
\end{verbatim}

Вищеприведене взаєморозміщення команд $\backslash$caption та $\backslash$label обов'язкове для провильної мітки при посиланні на рисунок.
