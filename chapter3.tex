\chapter{Підготовка демонстраційних прикладів та перспективи використання шаблону} 
\label{chap:first}

\section{Використання \LaTeX{} при оформленні робіт}

\subsection{Практика використання \LaTeX{} в науковій сфері}

Система \LaTeX{} у всьому світі широко використовується науковцями, особливо фахівцями з фундаментальних наук.

Основне призначення системи ~--- подготовка наукових документів (як правило, з технічних і фізико"=математичних наук). \LaTeX{} зручно використовувати для підготовки звітів з великою кількістю формул, таблиць та великим списко використаної літератури. Більшість наукових видань приймають тексти в форматі tex.

\subsection{Використання \LaTeX{} в ВНЗ}

Для фізико"=математичних спеціальностей актуальним є освоєння спеціалізованого програмного забезпечення для оформлення результатів дослідження. 

Крім того, що \LaTeX{} надає студентам потужний інструмент для оформлення результатів власних наукових робіт, він розвиває абстрактне мислення завдяки якістно відміному від загальноприйнятого у текстових процесорах WYSIWYG"=підходу до роботи з документом, адже написання и оформлення тексту і перегляд результату~--- це різні операції. Ще однією побічною перевагою використання \LaTeX{} у повсякденный роботі є вивчення базових прийомів програмування при написанні і використанні макросів.

В реаліях, де \LaTeX{} являється стандартом де"=факто для публікацій в технічних і фізико"=математичних журналах, його використання в вищих навчальних закладах при оформленні результатів студентських наукових робіт має підготувати майбутнього науковця до масштабної наукової роботи у майбутньому. В Україні численні внз вже рекомендують \LaTeX{} для оформлення курсових та випускних робіт. Серед них має сенс виділити національний технічний університет України <<Київський політехнічний інститут імені Ігоря Сікорського>>, де на кафедрі прикладної математики \LaTeX{} використовується для оформлення звітів до лабораторних робіт а на сайті кафедри електромеханічного обладнання енергоємних виробництв, крім того, для навчання студентів розміщені теоретичні та практичні завдання по роботі з пакетом \LaTeX{}.

\subsection{Переваги використання розробленого шаблону}


\section{Розробка інструкції та наочних прикладів по використанню шаблону}



