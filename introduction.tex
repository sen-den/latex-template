\likechapter{Вступ}

Науково-дослідна робота студентів є одним з найважливіших засобів формування майбутнього спеціаліста.

Завершальною частиною будь-якого наукового дослідження є літературне оформлення його матеріалів. При цьому, саме систематичність і доцільність оформлення презентуватимуть професіоналізм дослідника і пророблену ним роботу.

Найпростішою формою наукової роботи є реферат. Самостійним   дослідженням, що передбачає певний науково-практичний досвід студента є курсова робота. Дипломна робота є підсумком навчальної та  наукової діяльності студента у вищому навчальному закладі.

При написанні студентами наукових робіт, однією з задач, яка ставиться перед ними є оформлення тексту відповідно до виставлених кафедрою вимог, які, в цілому, базуються на вимогах ДСТУ 3008:2015 <<Інформація та документація. Звіти у сфері науки і техніки. Структура та правила оформлювання>>.

З розвитком Інтернету значного поширення набули електронні комунікації,тому електронні публікації зайняли вагому долю наукових публікацій в цілому. У видавництвах, які спеціалізуються на публікації наукової і технічної літератури міцні позиції займає видавнича система  \LaTeX{}, яка являється пакетом макросів для \TeX{} \cite{Knuth1984TheTeXbook}.

Пакет дозволяє автоматизувати значну кількість задач по підготовці наукових статей, серед яких формування змісту; нумерація заголовків всіх рівнів, формул, таблиць та ілюстрацій; розміщення ілюстрацій і таблиць на аркуші; ведення бібліографії тощо.

\textbf{Актуальність} роботи полягає в розповсюдженості видавничої системи при підготовці звітів про наукову роботу, до яких відносяться, зокрема курсові та випускні роботи студентів вищищих навчальних закладів.

\textbf{Ціллю} роботи над шаблоном є забезпечення максимальної відповідності отримуваного документу вимогам ДСТУ 3008:2015 і мінімізації затрат часу на оформлення роботи та дослідження вимог до нього.

\textbf{Об'єктом} проведеного дослідження є засоби реалізації вимог ДСТУ 3008:2015. 

\textbf{Предметом} роботи є шаблон для студентських наукових робіт, зокрема курсових та випускних робіт, що відповідає встановленим ДСТУ 3008:2015 вимогам, розроблений засобами системи \LaTeX{}.

Поставлено наступні \textbf{задачі}: 
\begin{enumerate}
\item дослідити вимоги ДСТУ 3008:2015 та методичні рекомендації Херсонського державного університету до оформлення курсових та випускних робіт, розглянути їх на предмет протирічь;
\item розробити шаблон засобами \LaTeX{}~--- новий тип документу, що забезпечить виконання наведених вимог;
\item підготувати розроблений шаблон до використання студентами університету~--- опублікуати його у відкритому доступі, розробити наочні приклади з використання окремих його частин та пояснення його роботи.
\end{enumerate}
