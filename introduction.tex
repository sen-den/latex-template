\likechapter{Вступ}

Науково-дослідна робота студентів є одним з найважливіших засобів формування майбутнього спеціаліста.

Завершальною частиною будь-якого наукового дослідження є літературне оформлення його матеріалів. При цьому, саме систематичність і доцільність оформлення презентуватимуть професіоналізм дослідника і пророблену ним роботу.

Найпростішою формою наукової роботи є реферат. Самостійним   дослідженням, що передбачає певний науково-практичний досвід студента є курсова робота. Дипломна робота є підсумком навчальної та  наукової діяльності студента у вищому навчальному закладі.

При написанні студентами наукових робіт, однією з задач, яка ставиться перед ними є оформлення тексту відповідно до виставлених кафедрою вимог, які, в цілому, базуються на вимогах ДСТУ 3008:2015 <<Інформація та документація. Звіти у сфері науки і техніки. Структура та правила оформлювання>>.

З розвитком Інтернету значного поширення набули електронні комунікації,тому електронні публікації зайняли вагому долю наукових публікацій в цілому. У видавництвах, які спеціалізуються на публікації наукової і технічної літератури міцні позиції займає видавнича система  \LaTeX{}, яка являється пакетом макросів для \TeX{} \cite{Knuth1984TheTeXbook}.

Пакет дозволяє автоматизувати значну кількість задач по підготовці наукових статей, серед яких формування змісту; нумерація заголовків всіх рівнів, формул, таблиць та ілюстрацій; розміщення ілюстрацій і таблиць на аркуші; ведення бібліографії тощо.

Для забезпечення максимальної відповідності отримуваного документу і мінімізації затрат часу на форматування роботи визначено необхідність розробки шаблону для студентських наукових робіт засобами видавничої системи \LaTeX{} та підготовки його до практичного використання.

Об'єктом проведеного дослідження є вимоги ДСТУ 3008:2015 і засоби реалізації цих вимог у системі \LaTeX{}. Предметом роботи є шаблон  для студентських наукових робіт, що відповідає встановленим вимогам. З розповсюдженості цієї видавничої системи при підготовці наукових публікацій слідує актуальність у її використанні при підготовці студентських наукових робіт та необхідність у забезпеченні відповідним шаблоном, як це прийнято в журналах наукових публікацій. 

Поставлено наступні задачі: дослідити вимоги ДСТУ 3008:2015 та методичні рекомендації Херсонського державного університету до оформлення курсових та випускних робіт, розглянути їх на предмет протирічь; розробити шаблон~--- новий тип документу, що забезпечить виконання наведених вимог; підготувати розроблений шаблон до використання студентами університету~--- опублікуати його у відкритому доступі, розробити наочні приклади та пояснення його роботи.

Основною ціллю роботи є підготока шаблону, який спростить роботу студентів над підготовкою результатів науковою роботи до публікації.