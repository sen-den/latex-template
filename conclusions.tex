\likechapter{Висновки}

В процесі розробки шаблону для студентських наукових робіт засобами системи \LaTeX{} розглянуто основні механізми роботи системи. Досліджено пакети, представлені на офіційному веб"=ресурсі \LaTeX{} \cite{www:ctan}, зроблено огляд можливостей з реалізаці та автоматизації роботи над документом, що надаються ними, обрано необхідні для реалізації шаблону пакети макросів.

Перед розробкою шаблону детально оглянуто вимоги ДСТУ 3008:2015 до оформлення звітів про наукову роботу в цілому та до окремих її частин, таких як <<Зміст>> та <<Титульний аркуш>>. Їх було порівняно з вимогами, пердставленими на офіційному сайті Херсонського державного університету, при цому було виявлено суттєві розбіжності з деяких питань.

За результатами проведеного порівняння було розроблено шаблон засобами системи \LaTeX{}, що задовільняє основним вимогам ДСТУ 3008:2015. При цьому окремо розглянуто додаткові рішення та недоліки, що вимагають доопрацювання.

Підготовлений шаблон для \LaTeX{} дозволяє автоматизуати значну кількість рутинних операцій при роботі над підготовкою звіту про результати студентської наукової роботи, а саме~--- курсової або випускної, серед яких форматування титульного аркушу, автоматична побудова зміту та списку використаної літератури, нумерація формул, таблиць, рисунків тощо. Для забезпечення доступу до завантаження шаблону його розміщено на популярному веб"=сервісі github.com за наступною адресою:

https://github.com/sen-den/latex-template/tree/publicTemplate.

В реаліях, де \LaTeX{} являється стандартом де"=факто для публікацій в технічних і фізико"=математичних журналах, його використання в вищих навчальних закладах при оформленні результатів студентських наукових робіт, а саме~--- курсових та випускних робіт, має підготувати майбутнього науковця до масштабної наукової роботи у майбутньому.

Крім того, що \LaTeX{} надає студентам потужний інструмент для оформлення результатів власних наукових робіт, він розвиває абстрактне мислення завдяки якістно відміному від загальноприйнятого у текстових процесорах WYSIWYG"=підходу до роботи з документом, адже написання и оформлення тексту і перегляд результату~--- це різні операції. Ще однією перевагою використання \LaTeX{} у повсякденній роботі є вивчення студентами базових прийомів програмування при написанні і використанні макросів.

В подальшому, можливою є розробка системи онлайн"=роботи з документацією, до якої входитиме розроблений шаблон. Така система дозволить мінімізувати час для вивчення нових і пошуку існуючих вимог до оформлення документів шляхом забезпечення користувачів готовим файлом"=шаблоном для системи \LaTeX{}, що надає актуальну версію оформлення.
