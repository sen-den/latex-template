\likechapter{Висновки}

В процесі розробки шаблону для студентських наукових робіт засобами системи \LaTeX{} розглянуто всі основні механізми системи. Поверхнево досліджено наповнення офіційного веб"=ресурсу \LaTeX{} \cite{www:ctan} та обрано необхідні для реалізації шаблону пакети макросів.

Перед розробкою шаблону детально досліджено вимоги ДСТУ 3008:2015 до оформлення звітів про наукову роботу в цілому та до окремих її частин, таких як <<Зміст>> та <<Титульний аркуш>>. Їх було порівняно з вимогами, пердставленими на офіційному сайті Херсонського державного університету, при цому було виявлено суттєві розбіжності з деяких питань.

За результатами проведеного порівняння було розроблено шаблон, що задовільняє основним вимогам ДСТУ 3008:2015. При цьому окремо розглянуто додаткові рішення та недоліки, що вимагають доопрацювання.

Підготовлений шаблон дозволяє автоматизуати значну кількість рутинних операцій при роботі над підготовкою звіту про результати наукової роботи, серед яких форматування титульного аркушу, автоматична побудова зміту та списку використаної літератури, нумерація формул, таблиць, рисунків тощо. Для забезпечення доступу до завантаження шаблону його розміщено на популярному веб"=сервісі github.com за адресою https://github.com/sen-den/latex-template/tree/publicTemplate.

В реаліях, де \LaTeX{} являється стандартом де"=факто для публікацій в технічних і фізико"=математичних журналах, його використання в вищих навчальних закладах при оформленні результатів студентських наукових робіт має підготувати майбутнього науковця до масштабної наукової роботи у майбутньому.

Крім того, що \LaTeX{} надає студентам потужний інструмент для оформлення результатів власних наукових робіт, він розвиває абстрактне мислення завдяки якістно відміному від загальноприйнятого у текстових процесорах WYSIWYG"=підходу до роботи з документом, адже написання и оформлення тексту і перегляд результату~--- це різні операції. Ще однією побічною перевагою використання \LaTeX{} у повсякденный роботі є вивчення студентами базових прийомів програмування при написанні і використанні макросів.

