\chapter{Огляд можливостей видавничої системи \LaTeX{} } 
\label{chapter:first}

\section{Принципи роботи \LaTeX{}}

\subsection{Структура проекту \LaTeX{}}

На   відміну   від   текстових   процесорів,   що   працюють   за   принципом WYSIWYG,  LaTeX,  не  маючи  графічного  інтерфейсу,  формує  результуючий документ з текстового файлу, що містить окремо власне текст і окремо інструкцію з його форматування, в термінах 
LaTeX --- преамбулу.

\subsection{Використання макророзширень в \LaTeX{}}

Основною  частиною  системи  LaTeX  є  велика  кількість  пакетів  макросів, кожен  з  яких  забезпечує  автоматизацію  і  полегшення  виконання  певних  дій  при створенні  документу.  На  момент  написання  статті  офіційний  веб-ресурс  LaTeX пропонує  5287  пакетів  макросів.  Так як  частина  з  них  надає  інструменти  для реалізації  одних  і  тих  самих  елементів,  то  була  поставлена  задача  обрати  серед них  необхідні  для  реалізації  всіх  поставлених  задач. 

Основною  частиною  системи  LaTeX  є  велика  кількість  пакетів  макросів, кожен  з  яких  забезпечує  автоматизацію  і  полегшення  виконання  певних  дій  при створенні  документу.  На  момент  написання  статті  офіційний  веб-ресурс  LaTeX пропонує  5287  пакетів  макросів.  Так як  частина  з  них  надає  інструменти  для реалізації  одних  і  тих  самих  елементів,  то  була  поставлена  задача  обрати  серед них  необхідні  для  реалізації  всіх  поставлених  задач. 

\section{Основні можливості \LaTeX{}}

\subsection{Базові види документів}

Вбудовані типи ипи документів LaTeX:
Звіт
Стаття
Книга
Лист

\subsection{Рубрикація документу}

Команди chapter, section, subsection...
Зміст

\subsection{Ведення бібліографії}
Проблема задачи.
Все автоматически.
БибТех.

\subsection{Автоматизовані процеси}

Пакет   дозволяє   автоматизувати   значну   кількість  задач  по  підготовці наукових статей,серед яких формування змісту; нумерація заголовків всіх рівнів, формул,  таблиць  та  ілюстрацій;  розміщення  ілюстрацій  і  таблиць  на  аркуші; ведення бібліографії тощо.
