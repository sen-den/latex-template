\chapter{Огляд можливостей видавничої системи \LaTeX{} } 
\label{chapter:first}

\section{Принципи роботи \LaTeX{}}

\subsection{Структура проекту \LaTeX{}}

На відміну від текстових процесорів, що працюють за принципом {WYSIWYG} (What You See Is What You Get~--- що бачиш, те ы отримуєш), \LaTeX{}, не маючи графічного інтерфейсу, формує результуючий документ з текстового файлу, що містить окремо власне текст і окремо інструкцію з його форматування, в термінах \LaTeX{} --- преамбулу.

Текст документу може розбиватися на декілька окремих файлів для полегшення роботи над частинами документу або розподілення її між кількома людьми. Крім того, в окремі файли виносяться бібліографічні посилання у форматі BibTeX та рисунки, що включаються до документу. 

Після завершення роботи над документом і запуском \LaTeX{}, він підключає вказані у преамбулі пакети макросів та опрацьовує файли проекту один, а за необхідності~--- декілька разів послідовно. При цьому, ним та супутніми програмами, такими як BibTeX, послідовно формуються тимчасові файли, що містять, наприклад, список бібліографії, посилань та змісту, а після~--- dvi (device indenendant) файл. Остайній придатний для перегляду та друку на будь"=якому комп'ютері з встановленим відповідним для нього dvi"=драйвером. При цьому гарантується однаковість форматування тексту на будь"=якому комп'ютері \cite[с.~16]{Lvovskii2010NaborVerstka}.

За необхідності, dvi"=файл може бути конвертований до інших форматів, серед яких~--- широко розповсюджений pdf \cite{Habr2012BacDiplom}\cite{Habr2012TempDisser}.

\subsection{Використання макророзширень в \LaTeX{}}

Основною частиною системи LaTeX є велика кількість пакетів макросів, кожен з яких забезпечує автоматизацію і полегшення виконання тих чи інших дій при створенні документу або вносить зміни в стандартні налаштування \LaTeX{}. На момент написання статті офіційний веб-ресурс LaTeX~\cite{www:ctan} пропонує 5287 пакетів макросів. 

Макрос представляє собою команду з назвою, що визначається або перевизначається наступними командами\cite{Sjutkin2002Manual}:
\begin{verbatim}
\newcommand{<назва команди>}[<кількісь параметрів>]{<тіло команди>}
\renewcommand{<назва команди>}[<кількісь параметрів>]{<тіло команди>}
\end{verbatim}

Назви всіх команд у \LaTeX{} починаються з символу зворотнього слешу <<$\backslash$>>. Кількість параметрів не може перевищувати 9. Для подолання обмеження можуть використовуватися пакети, наприклад, обраний xkeyval (таблиця \ref{table:packages}), що створюють власний механізм передачі аргументів в макрос. В тілі макросу отримані аргументи використовуються за своїм номером за порядком після символу <<\# >>.

Приклад простого макросу, що додає до документа перший параметр напівжирним а другий~--- курсивним шрифтом:
\begin{verbatim}
\newcommand{\boldAndItalik}[2]{\textbf{#1} \textit{#2}}
\end{verbatim}

Можливе використання попереднього макросу. На рисунку \ref{ris:image} зображено результат його роботи (збільшено).
\begin{verbatim}
\boldAndItalik{FirstText}{SecondText}
\end{verbatim}


\begin{figure}[ht]
\center{\includegraphics[width=1\linewidth]{img/makrosResultImg}}
\caption{Зразок роботи макросу}
\label{ris:image}
\end{figure}

Так як частина з існуючих пакетів макросів надає різні інструменти для реалізації одних і тих самих елементів, то було поставлено завдання обрати серед них необхідні для реалізації всіх поставлених задач з реалізації шаблону. Основні пакети, підключені до шаблону, перечислено в таблиці \ref{table:packages}

\begin{table}{|l|l|}{Основні підключені пакети}{table:packages}
	{\hline
	\parbox[t]{5cm}{Назва} & Опис \\
	\hline}
	fontenc & Дозволяє вказати кодування документа\\
	cmap & Забезпечує коректне кодування\\
	babel & Забезпечує необхідні мовні зміни\\
	geometry & Дозволяє змінити параметри сторінки\\
	indentfirst & Забезпечує відступ у першому абзаці\\
	hyperref & Позначає деякі елементі як гіперпосилання\\
	fancyhdr & Дозволяє налаштувати колонтитули\\
	titlesec & Дозволяє налаштувати вигляд заголовків\\
	titletoc & Дозволяє налаштувати вигляд змісту\\
	longtable & Надає потужніші таблиці і налаштування для них\\
	caption & Дозволяє налаштувати підписи різних елементів\\
	xkeyval & Надає новий спосіб передачі параметрів у макрос\\
	ifthen & Надає макрос\,--\,умовний оператор\\
\end{table}

При цьому, зроблений вибір ніяк не обмежує користувачів шаблону у підключенні інших, альтернативних або власних пакетів.

\subsection{Базові види документів}

Кожен документ в \LaTeX{} належить до одного з видів, <<класів>> в термінах \LaTeX{}, стандартних, що йдуть з самим \LaTeX{} або тих, що привносяться пакетами. В таблиці \ref{table:classes} наведено основні класи документів.

\begin{table}{|l|l|}{Основні вбудовані класи документів \LaTeX{}}{table:classes}
	{\hline
	\parbox[t]{5cm}{Назва} & Опис \\
	\hline}
	article & Статті для журналів, короткі звіти\\
	report & Великі звіти з кількох частин\\
	exreport & оновлений і розширений report\\
	book & Книга\\
	letter & Лист\\
	beamer & Презентація\\
\end{table}

Всі інші класи документів, що входять до складу пакетів або розроблюються автором самостійно для власних потреб, засновуються на одному з базових класів і зберігаються в окремому файлі з розширенням cls і іменем\,--\,назвою класу. Користувацькі класи підключаються до документу і працюють аналогічно базовим. 

\section{Основні можливості \LaTeX{}}

\subsection{Рубрикація документу}

Відповідно до вимог \cite{DSTU20153008}, документ розбивається на структурні частини, серед яких <<Вступ>>, <<Зміст>>, <<Висновки>>, <<Список використаних джерел>>, <<Додатки>> та, власне, текст документу, який теж розбивається на менші частини~--- розділи, підрозділи, пункти та підпункти. \LaTeX{} надає механізми для форматування заголовків всіх цих частин, проте їхнє овормлення не сповна відповідає накладеним вимогам. Їх опис та необхідні зміни описані в \ref{dev:toc}.

Зміст, як і деякі інші структурні частини документу, формується засобами \LaTeX{} автоматично з заголовків згідно вказаних налаштувань і заданого форматування. Детальніше внесені зміни розглянуто в \ref{dev:toc}.

Текстові процесори, найпопулярнішими серед яких є MS Word та OO Writer, надають досить потужні засоби для форматування та структурного розділення документу у вигляді стилів. Проте, все ще широко розповсюджене некоректне використання засобів текстових процесорів, призводить до того, що форматування до кожного елементу тексту авторами застосовується вручну. Це, як і стиль набору, який подекуди називають <<вирівнювання пробілами>>, дає в результаті документи низької якості, які складно піддаються відносно простим змінам. 

На рисунку \ref{ris:wrongFormat} наведено декілька типових помилок форматування: <<вирівнювання пробілами>> на титульному аркуші (a); <<зміст>>, сформований вручну, в якому, крім зайвих пробілів, присутнє порушення вимог до оформлення (c); заголовки, не сповна коректно оформлені, та абзацний відступ з пробілів; формула та її номер, вирівняні пробілами. 

\begin{figure}[h]
	\begin{minipage}[h]{0.47\linewidth}
		\center{\includegraphics[width=1\linewidth]{img/wrongFormatA}} a) \\
	\end{minipage}
	\hfill
	\begin{minipage}[h]{0.47\linewidth}
		\center{\includegraphics[width=1\linewidth]{img/wrongFormatB}} \\b)
	\end{minipage}
	\vfill
	\begin{minipage}[h]{0.47\linewidth}
		\center{\includegraphics[width=1\linewidth]{img/wrongFormatC}} c) \\
	\end{minipage}
	\hfill
	\begin{minipage}[h]{0.47\linewidth}
		\center{\includegraphics[width=1\linewidth]{img/wrongFormatD}} d) \\
	\end{minipage}
	\caption{Некоректне форматування документу в текстовому процесорі}
	\label{ris:wrongFormat}
\end{figure}

\LaTeX{} не дозволить використовувати прийоми, подібні наведеним на рисунку \ref{ris:wrongFormat}, для форматування тексту, що змусить автора користуватись коректними засобами форматування та, цим самим, підвищить якість готового до друку або перегляду документу.

\subsection{Ведення бібліографії}

Бібліографічні посилання в структурній частині <<Список використаних джерел>> мають наводитися відповідно до ДСТУ 7.1. Автоматична нумерація посилань і їх формування суттєво спрощують роботу над документом. 

Для роботи з бібліографічними посиланнями використовується BibTeX. Опрацьовуючи тимчасовий файл з посиланнями та текстовий файл з бібліографічними посиланнями у власному форматі, ним формується список використаної літератури, впорядкований за порядком першого входження цитат. Цей файл використовується \LaTeX{} при наступному опрацюванні документу.

Саме через те, що деякі зміни вносяться в тимчасові файли вже після опрацювання частини тексту, для отримання коректного документу (коректного за змістом, нумерацією бібліографії та посилань а не власне форматуванням) необхідно запускати \LaTeX{} декілька разів  (до чотирьх \cite{Stolyarov2010SverstaiDiplom}) поспіль.

\subsection{Автоматизовані процеси}

Пакет дозволяє автоматизувати значну кількість задач по підготовці наукових статей, серед яких вже згадані вище формування змісту; нумерація заголовків всіх рівнів, формул, таблиць та ілюстрацій; розміщення ілюстрацій і таблиць на аркуші; ведення бібліографії тощо.
